\documentclass[UTF8]{ctexart}
\usepackage{amsmath}
\title{第一章-习题3}
\author{姜帆}
\date{{\today}}
\begin{document}
\maketitle
%\tableofcontents
\section{第一题}
\begin{equation}
\begin{aligned}
\frac{d(R^{-1}p)}{dR}
&=\lim_{\phi \to 0}\frac{(Rexp(\phi^\wedge))^Tp-R^Tp}{\phi}\\
&=\lim_{\phi \to 0}\frac{(exp(\phi^\wedge))^TR^Tp-R^Tp}{\phi}\\
&=\lim_{\phi \to 0}\frac{(I+\phi^\wedge)^TR^Tp-R^Tp}{\phi}\\
&=\lim_{\phi \to 0}\frac{(\phi^\wedge)^TR^Tp}{\phi}\\
&=\lim_{\phi \to 0}\frac{(-\phi)^{\wedge}R^Tp}{\phi}\\
&=\lim_{\phi \to 0}\frac{R^T(-\phi)^{\wedge}p}{\phi}\\
&=\lim_{\phi \to 0}\frac{R^Tp^{\wedge}{(-\phi)}}{\phi}\\
&=R^Tp^\wedge
\end{aligned}
\end{equation}

\section{第二题}
\begin{equation}
\begin{aligned}
\frac{dln({R_1}{R^{-1}_2})}{dR_2}
&=\lim_{\phi \to 0}\frac{{ln({R_1{{R_2exp({\phi^\wedge)}^T}})}-{ln({R_1R^T_2})}}}{\phi}\\
&=\lim_{\phi \to 0}\frac{ln(R_1(exp(\phi^\wedge))^TR^T_2)-{ln({R_1R^T_2})}}{\phi}\\
&=\lim_{\phi \to 0}\frac{{ln(R_1R^T_2R_2(exp(\phi^\wedge))^TR^T_2)}-{ln({R_1R^T_2})}}{\phi}\\
&=\lim_{\phi \to 0}\frac{ln(R_1R^T_2(R_2exp(\phi^\wedge)R^T_2)^T)-ln({R_1R^T_2})}{\phi}\\
&=\lim_{\phi \to 0}\frac{ln(R_1R^T_2(exp((R_2\phi)^\wedge))^T)-ln(R_1R^T_2)}{\phi}\\
&=\lim_{\phi \to 0}\frac{ln(R_1R^T_2exp((-R_2\phi)^\wedge))-ln(R_1R^T_2)}{\phi}\\
&=\lim_{\phi \to 0}\frac{ln(R_1R^T)+J^{-1}_r(-R_2\phi)-ln(R_1R^T_2)}{\phi}\\
&=\lim_{\phi \to 0}\frac{-J^{-1}_r(R_2\phi)}{\phi}\\
&=-J^{-1}_r((ln(R_1R^{-1}_2))^\vee)R_2
\end{aligned}
\end{equation}
\end{document}
